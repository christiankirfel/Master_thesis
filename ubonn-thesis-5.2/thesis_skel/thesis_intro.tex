%==============================================================================
\chapter{Introduction}
\label{sec:intro}
%==============================================================================

The introduction usually gives a few pages of introduction to the
whole subject, maybe even starting with the Greeks.

For more information on \LaTeX{} and the packages that are available
see for example the books of Kopka~\cite{kopka04} and Goossens et
al~\cite{goossens04}.

A lot of useful information on particle physics can be found in the
\enquote{Particle Data Book}~\cite{pdg2010}.

I have resisted the temptation to put a lot of definitions into the
file \texttt{thesis\_defs.sty}, as everyone has their own taste as
to what scheme they want to use for names.
However, a few examples are included to help you get started:
\begin{itemize}
\setlength{\itemsep}{0pt}\setlength{\parskip}{0pt}
\item cross-sections are measured in \si{\pb} and integrated
  luminosity in \si{\invpb};
\item the \KoS is an interesting particle;
\item the missing transverse momentum, \pTmiss, is often called
  missing transverse energy, even though it is calculated using a vector sum.
\end{itemize}
Note that the examples of units assume that you are using the
\textsf{siunitx} package.

It also is probably a good idea to include a few well formatted
references in the thesis skeleton. More detailed suggestions on what
citation types to use can be found in the thesis
guide~\cite{thesis-guide}:
\begin{itemize}
\item articles in refereed journals\cite{pdg2010,Aad:2010ey};
\item a book~\cite{Halzen:1984mc};
\item a PhD thesis~\cite{tlodd:2012} and a Diplom thesis~\cite{mergelmeyer:2011};
\item a collection of articles~\cite{lhc:vol1};
\item a conference note~\cite{ATLAS-CONF-2011-008};
\item a preprint~\cite{atlas:perf:2009} (you can also use
  \texttt{@online} or \texttt{@booklet}for such things;
\item something that is only available online~\cite{thesis-guide}.
\end{itemize}

At the end of the introduction it is normal to say briefly what comes
in the following chapters.

The lines at the end of this file are used by AUCTeX to specify which
is the master \LaTeX{} file, so that you can compile your thesis
directly within \texttt{emacs}.

%%% Local Variables: 
%%% mode: latex
%%% TeX-master: "../mythesis"
%%% End: 
