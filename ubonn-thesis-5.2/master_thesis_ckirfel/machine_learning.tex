\section{Machine Learning}

The central aspect of this work is the us of machine learning in the field of High Energy Collider Physics, its difficulties and the steps of its fine tuning.\\
This chapter introduces Machine Learning as a tool for analysis. The concept of artificial neural networks as the implementation of artificial intelligence used in this work is introduced to motivate the main features of the approach later to be investigated in more detail.\\
The weakness of neural networks to systematic uncertainties is explained and the concept of an Adversarial neural network as suggested by Goodfellow is introduced.


\subsection{The concept of machine learning}

For many decades computers have been an integral aspect of science, handling large amounts of data, completing tedious calculations and controlling experiments. For the most part the  machines were assigned discrete tasks. They stepwise followed commands previously designed by human users and with non statistical outcomes. 
For particle physics in particular computers were used to select interesting data from large samples when given proper instructions making it possible to process amounts of data that would have exceeded human capacities. However the selection rules had always to be generated by the user, therefore requiring an in depth analysis of the underlying system. Artificial intelligence presents a way for a program to establish its own decision rules and improve these over several iterations and thereby learning to solve the problem all by itself.\\
There has been a great effort over the last decades trying to implement a way for machines to learn from experience and thereby enable them to analyze complex tasks in a wide range of applications.\\
Often times the human understanding of the task and the work done by the machine itself are in equilibrium as great knowledge of an assignment is required to find the most efficient way to have a machine not only work on it but work on it efficiently. This means fine tuning the machine's capabilities and degrees of freedom to the requirements and is called hyperparamter optimization.\\
The task of machine learning can be simplified by comparing what a living being needs to be able to learn solving new problems, which can be split into understanding a system, evaluating a decision step and generating new decision steps.\\
Understanding a system means when given a task having a way to observe all features and possibilities relevant to work on the task. Humans have their senses to easily break down their observations into useful features and concepts that can then be processed for decision making. A computer has no senses built in and for most tasks this means that the step of filtering information fo a relevant subset of features has still to be done by humans or a good preprocessing algorithm.\\
Once a system has been converted to a subset of features usable by a computer the step of making own decisions has to be implemented. This can be done by weighting and interconnecting the information using structures often times inspired by the structure of the neurons in the human brain.\\
This process of decision making at the basis is somewhat random and so far the network has no way of evaluating the success of a decision rule. This is where a cost function is introduced that pairs a generated decision rule with an estimator of its quality. Paired with an optimizer for the next step this function is a basis for a network to independently approach a good decision rule for a somewhat un-investigated topic.\\
A very commonly used machine learning algorithm of the artificial neural network which on its own forms a quite broad field that builds the base of this work. The most important concepts of machine learning will be explained using the example of a neural network.

\subsection{Neural Networks}

The artificial neural network, or just neural network, is the most commonly known approach to machine learning. Its structure is inspired by the neurons forming the human brain which is also where it gets its name from.\\
Instead of neurons a neural network consists of numerous very simple processors, called nodes. These nodes are usually structured into several layers. As presented in fig 1. Also there is several ways to structure and connect these nodes often times matching a certain problem in this explanation only the most commonly way is explained. In that case each node of a layer is getting input from each node in the last layer and is outputting to each node of the following layer.\\

\subsubsection{The input layer}

To understand a task and draw reasonable conclusions first the underlying system has to be understood. Its features need to be found and summarized. The human brain is capable of investigate unknown systems and learn the features that are the most unique or interesting. For that it has its senses to explore the system and process them later. To allow a machine to do something similar the unknown system has to be represented in a way that it is clear for the network what to look out for. This is usually the task that needs the most preprocessing by the user. The simplest case is to submit a list of variables to the network. In particle physics this could be kinematic variables of the particles in an event.
The input to a neural network is just given to the input layer of nodes and then processed through each following layer. For different tasks different layers might deal with different parts of the information but in this work the linear way of giving all information used to an input layer and then processing it is used.

\subsubsection{Decision making process}


The second problem is the definition of a clear task. A network could for example be used to generate new features or to distinguish certain features. In any way it has to be defined what the network is supposed to do.
The last part is to give a network a way to measure how well it is doing at out task. Comperable to our learning experience.
In the end it all comes down to finding smart mathematical representations of a system and and of learing success.
A possible approach for machine learning is the neural network. The following chapter will explain the idea of neural networks and their main features. In addition the adversarial neural network used for this research will be introduced in detail. 

\subsection{Adversarial Neural Networks}

\subsection{Hyperparamters}

nodes,layers, learning rate, optimizer, momentum, loss function

Convolutional Dense Linear

