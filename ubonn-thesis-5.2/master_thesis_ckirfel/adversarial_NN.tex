\chapter{Adversarial Neural Network}
\label{chp:ANN}

In this chapter the setup and training of the Adversarial Neural Network is described. The Network did not achieve the desired results in its initial configuration.
For this reason different configurations for the implementation of the network structure are presented in addition to the hyperparamter investigations.
The first part of the chapter focuses on the initial run of the ANN using the base network presented in chapter~\ref{chp:simple_NN}. The results are investigated and the hyperparameters are adapted using an initial setup for the second network.
The second network is investigated trying out a number of significantly differetn setups.

Furthermore the problems with this setup are discussed and used to motivate a different approach adapting the network input which is then also investigated.
Lasttly the effectiveness of the two models is discussed presenting possible future steps as the training and setup were strongly limited by the time constraint of this work.
\section{Setup of the first network}

The adversarial setup uses the classifier trained and tunes in chapter~\ref{chp:simple_NN} as a basis. For the classical setup this output is used as input for the adversarial network.
To tune the hyperparameters of the second network the classifier was left unchanged. 

\section{Choice of architecture}

\section{Optimisation setup}

\section{}
