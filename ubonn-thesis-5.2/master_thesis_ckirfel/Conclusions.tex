\chapter{Conclusions}

The work presented in this thesis focussed on two issues of neural networks. The first one concerns the behaviour of neural networks and adversarial neural networks for classification tasks in particle physics. Here, the influence of the hyper-parameters on network performance for different setups was tested in order to understand how a classifier should be set up and how it can be upgraded to an adversarial neural network.

The second one addresses the possibilities offered by adversarial neural networks to limit the influence of systematic uncertainties on a classification model. In particular, the technique of an adversarial network was tested on the \tW-\ttbar separation and the systematic uncertainty arising from the different simulation approaches of DR and DS for the interference term at NLO.

A fundamental knowledge of neural networks and a set of hyper-parameters were discussed. It was shown that a good set of parameters is not only dependent on the topology of the problem but also massively depends on the setup of the network structure. A good classifier does not generally also form a good basis for an adversarial network structure. Instead, an adversarial network requires a low learning rate and above all a slow optimisation because more features of the problem need to be taken into account. This is due to the sensitivity to a certain systematic being added to the primary task of classification. The initial optimum, established for the classifier alone, that allowed for straightforward and fast training is not the desired model anymore.

Moreover, the adversarial neural network tested in this work is relatively unstable and the outcome changes significantly when hyper-parameters are slightly adjusted. This effect was not observed for the classifying network and resulted in a laborious optimisation process, which made the constraint of computational power even more significant.
The most important feature of the adversarial network, investigated with the three approaches presented in chapter~\ref{chp:ANN}, is the dimension of input it is provided with. There is a strong argument that only a single node, as classically suggested, does not justify a complex adversarial network architecture. In addition to that, the general information lost when looking only at the final node should not be neglected during the setup. The performances achieved were strongly dependent on the shape of input.

Lastly, a significant improvement on the sensitivity of the classifier for the systematic sample could not be achieved. Although the classifier was able to generate a model that rendered its adversary unable to learn, no significant differences were visible in the distribution. The shapes changed but, at the stage of this work, no sophisticated argument can be made for an improvement. In conclusion, switching from the single-node-input to a more complex input for the adversary seems to be the more promising approach. Combined with a different set of variables and a detailed hyper-parameter scan one might be able to achieve a reduction of sensitivity while keeping the quality of the overall classification constant.
For similar problems it would be interesting to test out the consequences of only providing the data actually affected by the systematic uncertainties to the adversary. In this work the systematics only affected the \tW-channel and the \ttbar was still considered nominal. This could have a strong bias effect on the result.

Neural networks in general are a promising technique for researchers in particle physics. The variety of highly different structures and hyper-parameters of the networks make them a potentially powerful tool for different scopes of research. An improvement in the sensitivity to systematics has been achieved for different researches using adversarial network approaches. Although a small effect on the sensitivity was achieved in this work, the sensitivity was not reduced as far as hoped for. Perhaps the topology of the problem and the shape of the systematics mean that the task has no model that is insensitive. However, the strategies presented in this work, pave the way for further research focusing on the approach and the appliance of machine learning techniques for data analysis in particle physics.

