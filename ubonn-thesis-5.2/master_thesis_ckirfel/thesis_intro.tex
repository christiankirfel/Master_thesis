%==============================================================================
%This chapter should motivate why I did this work
%This chapter should introduce PP and ML and connect them
%It should outline what I do in this thesis and why


\chapter{Introduction}
\label{sec:intro}
One of my predominant memories from my childhood is Lego. Lego is and was a system of tools that aloud to build a variety of structures from castles to space ships using a variety of stones. The amount of pieces available has grown over the years and that at least in my eyes cost Lego a lot of its original charm. The fascination for a system enabling the user to build almost everything using a very simple set of elemental pieces has stayed with me over the years. Its simplicity and tidiness was not only appealing but also inspiring to me. 
And it was this fascination I was reminded of when I first learned about particle physics and the standard model of the same. The standard model being a fascinating approach that promised to boil down physics to a surprisingly simple set of pieces and interactions connecting them.\\
Being allowed to learn about the research process of this field has therefor been a great deal for me.\\
This work focuses on 
When it comes to researching at the level of particle physics the usual approach is to generate energy high enough to create new particles. Unfortunately this is a statistic process and high energies and high event count make it an enormous challenge to separate the interesting events from those of no interest. While for some time making cuts and designing detectors in a smart way was sufficient to achieve a high efficiency machine elarning is more and more relevant.
In this work the algorithm of an Adversarial Network is used to eparate the tw from the tt bar chanel. The tuning and step by step development of the network strcuture is described to give an outlne of the difficulties to overcome.
